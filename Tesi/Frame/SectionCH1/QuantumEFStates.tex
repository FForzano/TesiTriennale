\section{QEF States}
    In this section, some quantum states of electromagnetic field useful for quantum 
    communication are characterized. A brief introduction to another one tool for the 
    description of quantum systems is initially given, so some Gaussian and 
    non-Gaussian states are characterized.

    \subsection{Phase-space description}
        As seen before, in \ref{eq:QEF.4}, quantum system can be completely
        described by a density operator $\Xi$ defined in an infinite-dimensional Hilbert space
        $\Hilbert$. This operator can be expressed by the Fock representation (\ref{def:3}).
        Sometimes, however, it is convenient to give another representation of state $\Xi$ by
        means of a complex function introduced by Wigner \cite{Wigner}: the quasi-probability 
        distribution. In this thesis, this representation will be introduced and it will be used to
        classify the possible states.

        \begin{definition}[Quantum characteristic function]
            The s-order characteristic function $\mathcal{\chi}(\xi,s)$, with $\xi,s\in\mathbb{C}$,
            associated to the quantum state $\pmb{\Xi}$ is defined as:
            \begin{equation}
                \mathcal{\chi}(\xi,s)=\exp{\frac{s}{2}\absolutevalue{\xi}^2}
                tr\{\pmb{\Xi}\pmb{D}_\xi\}
            \end{equation}
            where $\pmb{D}_\xi$ is  the displacement operator of parameter $\xi$, defined as:
            \begin{equation}
                \pmb{D}_\xi=\exp{\xi\pmb{A}^\dagger-\xi^*\pmb{A}}.
            \end{equation}
            \label{def:QEFStates.1}
        \end{definition}
        \begin{definition}[Quasi-probability distribution]
            The s-order quasi-probability distribution $W(\alpha,s)$, with $s\in\mathbb{C}$,
            associated to the quantum state $\mathbf{\Xi}$ is given by:
            \begin{equation}
                W(\alpha,s)=\frac{1}{\pi^2}\int_{\mathbb{R}^2} 
                \mathcal{\chi}(\xi,s)e^{\alpha\xi^*-\alpha^*\xi}d\xi^2.
            \end{equation}
            \label{def:QEFStates.2}
        \end{definition}
        The quasi-probability distribution, for $s=0$ ($W(\alpha)=W(\alpha,0)$) is called 
        Wigner W-function.
        
    \subsection{Gaussian states}
    With the Wigner W-function $W(\alpha)$, it is possible to define the concept of Gaussian state
    (\cite{tesiGuerrini} quoting \cite{Gaussian1,Gaussian2,Gaussian3,Gaussian4,Gaussian5}).
    \begin{definition}[Gaussian state]
        A quantum state $\pmb{\Xi_G}$ is a Gaussian state if its Wigner W-function $W_G(\alpha)$
        is Gaussian, i.e
        \begin{equation}
            W_G(\alpha)=\frac{1}{\pi\sqrt{det\check{\pmb{C}_0}}}\exp{
                -\frac{1}{2}(\check{\pmb{\alpha}}-\check{\pmb{\mu}})^H
                \check{\pmb{C}_0}^{-1}(\check{\pmb{\alpha}}-\check{\pmb{\mu}})}.
        \end{equation}
        where $\check{\pmb{\mu}} $ is the augmented displacement vector, and $\check{\pmb{C}}_0$ 
        is the augmented covariance matrix.
    \end{definition}

    \paragraph{Coherent states}\mbox{} \\
        A coherent state is the state of a quantum armonic oscillator of amplitude $\mu$.
        It is defined (\cite{tesiGuerrini} seen \cite{CohSt_Glauber,CohSt_Glauber2}) as the eigenvector $\ket{\mu}$ of $\pmb{A}$ 
        associated to the eigenvalue $\mu$; i.e
        \begin{equation}
            \pmb{A}\ket{\mu}=\mu\ket{\mu}.
        \end{equation}
        It is possible to obtain a coherent state of parameter $\mu$, from the ground state as
        \begin{equation}
            \ket{\mu}=\pmb{D}_\mu\ket{0}.
        \end{equation}
        As mentioned before, it is possible to characterize a state with the Fock representation
        and, equivalently, with the Wigner W-function. The last one is given, for a coherent state,
        by \cite{QuantumNoise}:
        \begin{equation}
            W(\alpha)=\frac{2}{\pi}\exp{-2\absolutevalue{\alpha-\mu}^2}.
        \end{equation}
        It is easy to proof that $W(\alpha)$ is gaussian, with $\check{\pmb{\mu}}=[\mu\ \mu^*]^T$ and
        \begin{equation*}
            \check{\pmb{C}}_0=\frac{1}{2}\pmb{I}.
        \end{equation*} 
        The Fock representation is given by \cite{Dowling}:
        \begin{equation}
            \ket{\mu}=e^{-\frac{\absolutevalue{\mu}}{2}^2}\sum_{n=0}^{\infty}
            \frac{\mu^n}{\sqrt{n}}\ket{n}.
        \end{equation}

    \paragraph{Noisy coherent states}\mbox{} \\
        It is possible to characterize the state of a noisy armonic oscillator introducing
        the thermal state, i.e the state of a electromagnetic cavity system.
        The Fock representation of the thermal state $\pmb{\Xi}_{th}$ is given by \cite{tesiGuerrini}
        \begin{equation}
            \pmb{\Xi}_{th}=(1-v)\sum_{n=0}^{\infty}v^n\ket{n}\bra{n}
        \end{equation}
        where
        \begin{equation*}
            v=\frac{\bar{n}}{\bar{n}+1}
        \end{equation*}
        and $\bar{n}$ is the well-known Plank distribution
        \begin{equation*}
            \bar{n}=\left(\exp{-\frac{\hbar\omega}{k_B T}-1}\right)^{-1}.
        \end{equation*}

        A noisy coherent states $\pmb{\Xi}_{th}(\mu)$ of parameter $\mu$ can be obtained by 
        appling the displacement operator $\pmb{D}_\mu$ to the thermal state $\pmb{\Xi}_{th}$,
        as follow:
        \begin{equation}
            \pmb{\Xi}_{th}(\mu)=\pmb{D}_\mu^\dagger \pmb{\Xi}_{th} \pmb{D}_\mu.
        \end{equation}
        The Wigner W-function is given by
        \begin{equation}
            W_{th}(\alpha)=\frac{1}{\pi(\bar{n}+\frac{1}{2})}\exp{-\frac{\absolutevalue{\alpha-\mu}^2}
            {\bar{n}+\frac{1}{2}}}
        \end{equation}
        and it can be proved that it is a Gaussian function with $\check{\pmb{\mu}}=[\mu\ \mu^*]^T$
        and
        \begin{equation*}
            \check{\pmb{C}}_0=\left(\bar{n}+\frac{1}{2}\right)\pmb{I}.
        \end{equation*}
        The Fock representation is given by
        \begin{equation}
            \bra{n}\pmb{\Xi}_{th}(\mu)\ket{m}=(1-v)e^{-(1-v)\absolutevalue{\mu}^2}\sqrt{\frac{n!}{m!}}
            v^n[(1-v)\mu^*]^{m-n}L_n^{m-n}\left(\frac{-(1-v)^2\absolutevalue{\mu}^2}{v}\right)
            \label{eq:FRCS}
        \end{equation}


    \paragraph{Squeezed states}\mbox{} \\
    A squeezed state is a 

    \paragraph{Noisy squeezed states}\mbox{} \\

    \subsection{Non-Gaussian states}

    \paragraph{Photon added coherent states}\mbox{} \\

    \paragraph{Photon added squeezed states}\mbox{} \\