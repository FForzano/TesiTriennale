\section{QEF States}
    This section characterizes some quantum states of electromagnetic field. These states are
    helpful for the development of quantum communication and, in a more general way, of quantum
    networks. It is firstly given a brief introduction to another one tool involved in the 
    description of quantum systems. Secondly, some Gaussian and non-Gaussian states 
    are characterized.

    \subsection{Phase-space description}
        As seen before in \ref{eq:QEF.4}, a quantum system can be completely
        described by a density operator $\pmb{\Xi}$ defined in an infinite-dimensional Hilbert space
        $\Hilbert$, and this operator can be expressed by the Fock representation (\ref{def:3}).
        Sometimes, however, it is convenient to give another representation of state $\pmb{\Xi}$ by
        means of a complex function introduced by Wigner \cite{Wigner}: the quasi-probability 
        distribution. In this thesis, this representation will be introduced and it will be used to
        classify the possible states.

        \begin{definition}[Quantum characteristic function]
            The s-order characteristic function $\mathcal{\chi}(\xi,s)$, with $\xi,s\in\mathbb{C}$,
            associated to the quantum state $\pmb{\Xi}$ is defined as:
            \begin{equation}
                \mathcal{\chi}(\xi,s)=\exp{\frac{s}{2}\absolutevalue{\xi}^2}
                tr\{\pmb{\Xi}\pmb{D}_\xi\}
            \end{equation}
            where $\pmb{D}_\xi$ is  the displacement operator of parameter $\xi$, defined as:
            \begin{equation}
                \pmb{D}_\xi=\exp{\xi\pmb{A}^\dagger-\xi^*\pmb{A}}.
            \end{equation}
        \end{definition}
        \begin{definition}[Quasi-probability distribution]
            The s-order quasi-probability distribution $W(\alpha,s)$, with $s\in\mathbb{C}$,
            associated to the quantum state $\mathbf{\Xi}$ is given by:
            \begin{equation*}
                W(\alpha,s)=\frac{1}{\pi^2}\int_{\mathbb{R}^2} 
                \mathcal{\chi}(\xi,s)e^{\alpha\xi^*-\alpha^*\xi}d\xi^2.
            \end{equation*}

        \end{definition}
        The quasi-probability distribution, for $s=0$ ($W(\alpha)=W(\alpha,0)$) is called 
        Wigner W-function.
        
    \subsection{Gaussian states}
    With the Wigner W-function $W(\alpha)$, it is possible to define the concept of Gaussian state
    (\cite{tesiGuerrini} quoting \cite{Gaussian1,Gaussian2,Gaussian3,Gaussian4,Gaussian5}).
    \begin{definition}[Gaussian state]
        A quantum state $\pmb{\Xi_G}$ is a Gaussian state if its Wigner W-function $W_G(\alpha)$
        is Gaussian, i.e
        \begin{equation}
            W_G(\alpha)=\frac{1}{\pi\sqrt{det\check{\pmb{C}_0}}}\exp{
                -\frac{1}{2}(\check{\pmb{\alpha}}-\check{\pmb{\mu}})^H
                \check{\pmb{C}_0}^{-1}(\check{\pmb{\alpha}}-\check{\pmb{\mu}})}.
        \end{equation}
        where $\check{\pmb{\mu}} $ is the augmented displacement vector, and $\check{\pmb{C}}_0$ 
        is the augmented covariance matrix.
        \label{def:Gaussian}
    \end{definition}

    \paragraph{Coherent states}\mbox{} \\
        A coherent state is the state of a quantum armonic oscillator of amplitude $\mu$.
        It is defined (\cite{tesiGuerrini} seen \cite{CohSt_Glauber,CohSt_Glauber2}) as the eigenvector $\ket{\mu}$ of $\pmb{A}$ 
        associated to the eigenvalue $\mu$; i.e
        \begin{equation}
            \pmb{A}\ket{\mu}=\mu\ket{\mu}.
        \end{equation}
        It is possible to obtain a coherent state of parameter $\mu$, appliying the displacement
        operator to the ground state:
        \begin{equation}
            \ket{\mu}=\pmb{D}_\mu\ket{0}.
        \end{equation}
        As mentioned before, it is possible to characterize a state with the Fock representation
        and, equivalently, with the Wigner W-function. The last one is given, for a coherent state,
        by \cite{QuantumNoise}:
        \begin{equation}
            W(\alpha)=\frac{2}{\pi}\exp{-2\absolutevalue{\alpha-\mu}^2}.
        \end{equation}
        It is easy to proof that $W(\alpha)$ is gaussian, with $\check{\pmb{\mu}}=[\mu\ \mu^*]^T$ and
        \begin{equation*}
            \check{\pmb{C}}_0=\frac{1}{2}\pmb{I}.
        \end{equation*} 
        The Fock representation is given by \cite{Dowling}:
        \begin{equation}
            \ket{\mu}=e^{-\frac{\absolutevalue{\mu}}{2}^2}\sum_{n=0}^{\infty}
            \frac{\mu^n}{\sqrt{n}}\ket{n}.
        \end{equation}

    \paragraph{Noisy coherent states}\mbox{} \\
        \label{par:cohState}
        It is possible to characterize the state of a noisy armonic oscillator introducing
        the thermal state, i.e the state of an electromagnetic cavity in thermal equilibrium.
        The Fock representation of the thermal state $\pmb{\Xi}_{th}$ is given by \cite{tesiGuerrini}
        \begin{equation}
            \pmb{\Xi}_{th}=(1-v)\sum_{n=0}^{\infty}v^n\ket{n}\bra{n}
        \end{equation}
        where
        \begin{equation*}
            v=\frac{\bar{n}}{\bar{n}+1}
        \end{equation*}
        and $\bar{n}$ is the well-known Plank distribution
        \begin{equation*}
            \bar{n}=\left(\exp{\frac{\hbar\omega}{k_B T}}-1\right)^{-1}.
        \end{equation*}

        A noisy coherent states $\pmb{\Xi}_{th}(\mu)$ of parameter $\mu$ can be obtained by 
        appling the displacement operator $\pmb{D}_\mu$ to the thermal state $\pmb{\Xi}_{th}$,
        as follow:
        \begin{equation}
            \pmb{\Xi}_{th}(\mu)=\pmb{D}_\mu^\dagger \pmb{\Xi}_{th} \pmb{D}_\mu.
        \end{equation}
        The Wigner W-function is given by \cite{QuantumNoise}
        \begin{equation}
            W_{th}(\alpha)=\frac{1}{\pi(\bar{n}+\frac{1}{2})}\exp{-\frac{\absolutevalue{\alpha-\mu}^2}
            {\bar{n}+\frac{1}{2}}}
        \end{equation}
        and it can be proved that it is a Gaussian function with $\check{\pmb{\mu}}=[\mu\ \mu^*]^T$
        and
        \begin{equation*}
            \check{\pmb{C}}_0=\left(\bar{n}+\frac{1}{2}\right)\pmb{I}.
        \end{equation*}
        The Fock representation is given by
        \begin{equation}
            \bra{n}\pmb{\Xi}_{th}(\mu)\ket{m}=(1-v)e^{-(1-v)\absolutevalue{\mu}^2}\sqrt{\frac{n!}{m!}}
            v^n[(1-v)\mu^*]^{m-n}L_n^{m-n}\left(\frac{-(1-v)^2\absolutevalue{\mu}^2}{v}\right)
            \label{eq:FRCS}
        \end{equation}


    \paragraph{Squeezed states}\mbox{} \\
    \label{squeezedStates}
    A squeezed state with amplitude $\mu$ and squeezing parameter $\zeta$, is a defined as 
    \cite{tesiGuerrini,YuenRadField,QMnoiseInterf}
    \begin{equation}
        \ket{\mu,\zeta}=\pmb{D}_\mu\pmb{S}_\zeta\ket{0}
    \end{equation}
    where $\pmb{S}_\zeta$ is the squeezing operator, defined as
    \begin{equation}
        \pmb{S}_\zeta=\exp{\frac{1}{2}\left(\zeta\left(\pmb{A}^\dagger\right)^2+
        \zeta^*\pmb{A}^2\right)}.
    \end{equation}
    It can be proven that a squeezed state is a Gaussian state with $\check{\pmb{\mu}}=[\mu\ \mu^*]^T$
    and
    \begin{equation*}
        \check{\pmb{C}}_0=\frac{1}{2}
        \begin{bmatrix}
            \cosh(2r) && \sinh(2r)e^{-i\phi}\\
            \sinh(2r)e^{-i\phi} && \cosh(2r)
        \end{bmatrix}
    \end{equation*}
    with $\zeta=re^{i\phi}$.
    The Wigner W-function of a squeezed state, differently from the one of a coherent state, has not a 
    circular symmetry.

    \paragraph{Noisy squeezed states}\mbox{} \\
    The representation of a noisy squeezed state $\pmb{\Xi}_{th}(\mu,\zeta)$ is obtained,
    similarly to a noisy coherent state, as:
    \begin{equation}
        \pmb{\Xi}_{th}(\mu,\zeta)=\pmb{D}_\mu\pmb{S}_\zeta\pmb{\Xi}_{th}
        \pmb{S}_\zeta^\dagger\pmb{D}_\mu^\dagger.
    \end{equation}
    The Gaussian Wigner W-function is obtained with $\check{\pmb{\mu}}=[\mu\ \mu^*]^T$ and
    \begin{equation*}
        \check{\pmb{C}}_0=\left(\bar{n}+\frac{1}{2}\right)
        \begin{bmatrix}
            \cosh(2r) && \sinh(2r)e^{-i\phi}\\
            \sinh(2r)e^{-i\phi} && \cosh(2r)
        \end{bmatrix}.
    \end{equation*}
    The Fock representation is given by \cite{MarMar_1993}
    \begin{equation}\begin{split}
        \bra{n}\pmb{\Xi}_{th}(\mu,\zeta)\ket{m}=\frac{\pi Q(0)}{(n!m!)^{1/2}}
        \sum_{k=0}^{min(n,m)} k! \binom{n}{k}\binom{m}{k} \tilde{A}^k\left(\frac{1}{2}
        \tilde{B}\right)^{(n-k)/2}\\ \left(\frac{1}{2}\tilde{B}^*\right)^{(m-k)/2}
        H_{n-k}((2\tilde{B})^{-1/2}\tilde{C}) H_{m-k}((2\tilde{B}^*)^{-1/2}\tilde{C}^*) 
        \label{FR_NSS}
    \end{split}\end{equation}
    where $H_n$ is the Hermite polynomial with parameter $n$,
    \begin{equation*}
        Q(0)=\frac{1}{\pi}[(1+A)^2-\absolutevalue{B}^2]^{-1/2}\exp{-\frac{(1+A)\absolutevalue{C}^2
        +\frac{1}{2}[B(C^*)^2+B^*C^2]}{(1+A)^2-\absolutevalue{B}^2}},
    \end{equation*} 
    \begin{equation*}
        \tilde{A}=\frac{A(1+A)-\absolutevalue{B}^2}{(1+A)^2-\absolutevalue{B}^2},
    \end{equation*}
    \begin{equation*}
        \tilde{B}=\frac{B}{(1+A)^2-\absolutevalue{B}^2},
    \end{equation*}
    \begin{equation*}
        \tilde{C}=\frac{(1+A)C+BC^*}{(1+A)^2-\absolutevalue{B}^2};
    \end{equation*}
    and
    \begin{equation*}
        A=\bar{n}+(2\bar{n}+1)(\sinh(r))^2,\ 
        B=-(2\bar{n}+1)e^{i\phi}\sinh(r)\cosh(r),\ 
        C=\mu.
    \end{equation*}

    \subsection{Non-Gaussian states}
    A state that does not fulfill the definition \ref{def:Gaussian} is a non-Gaussian state.
    An important and useful for communications class of non-Gaussian states, are the photon 
    added states, examined in this thesis.

    \paragraph{Photon added states}\mbox{} \\
        \label{PAS}
        The photon added state $\pmb{\Xi}^{(1)}$, obtained from the quantum state $\pmb{\Xi}$,
        is given by:
        \begin{equation}
            \pmb{\Xi}^{(1)}=\frac{\pmb{A}^\dagger\pmb{\Xi}\pmb{A}}
            {tr\{\pmb{A}^\dagger\pmb{\Xi}\pmb{A}\}}.
        \end{equation}
        The name \emph{photon addition} could lead to believe that the mean photon number of the 
        photon added state is icreased by one compared to the previous non photon added state.
        However, that is incorrect.
        In general, its mean number of photons could be the same, more or less than the starting state.
        Only if $\pmb{\Xi}=\ket{n}\bra{n}$, i.e $\pmb{\Xi}$ is the density operator corresponding to
        the Fock state $\ket{n}$, the result of the photon addition is a state with one more photon.

        Logically, the photon added state $\pmb{\Xi}^{(k)}$ (with $k$ photon addition) is given by
        \begin{equation}
            \pmb{\Xi}^{(k)}=\frac{(\pmb{A}^\dagger)^k\pmb{\Xi}\pmb{A}^k}
            {tr\{(\pmb{A}^\dagger)^k\pmb{\Xi}\pmb{A}^k\}}.
        \end{equation}
        The Fock representation of a photon added state, can be obteined as:
        \begin{equation}
            \pmb{\Xi}^{(k)}=\frac{\tilde{\pmb{\Xi}}^{(k)}}{tr\{\tilde{\pmb{\Xi}}^{(k)}\}}
        \end{equation}
        and
        \begin{equation*}
            \bra{n}\tilde{\pmb{\Xi}}^{(k)}\ket{m}=
            \begin{cases}
                \sqrt{\frac{n!m!}{(n-k)!(m-k)!}}\bra{n-k}\pmb{\Xi}\ket{m-k}\ if\ n,m \geq k\\
                0\ otherwise
            \end{cases}
        \end{equation*}
        If $\pmb{\Xi}$ is a noisy coherent state of amplitude $\mu$ $(\pmb{\Xi}=\pmb{\Xi}_{th}(\mu))$,
        the photon added state $\pmb{\Xi}_{th}^{(k)}(\mu)$ is called noisy photon added coherent state
        (PACS); if $\pmb{\Xi}$ is a noisy squeezed state with amplitude $\mu$ and squeezing factor 
        $\zeta$ $(\pmb{\Xi}=\pmb{\Xi}_{th}(\mu,\zeta))$, the photon added state $\pmb{\Xi}_{th}^{(k)}
        (\mu,\zeta)$ is called noisy photon added squeezed state (PASS).