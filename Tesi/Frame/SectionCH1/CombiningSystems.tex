\section{Combining Systems}
The last postulate \ref{post:6} has very important consequences for composite system. It is possible
to describe two tipes of combined systems:

    \begin{definition}[Product states]
        A state $\ket{\psi}\in\Hilbert$ with $\Hilbert=\Hilbert_1\otimes\Hilbert_2$ is a
        pure state if exists $\ket{\psi_1}\in\Hilbert_1$ and $\ket{\psi_2}\in\Hilbert_2$
        such that:
        \begin{equation*}
            \ket{\psi}=\ket{\psi_1}\otimes\ket{\psi_2}.
        \end{equation*}
        \label{def:1}
    \end{definition}
    A product state represents two states which do not interact; an operation on one of 
    them does not perturb the other.

    \begin{definition}[Entengled states]
        A system that is not in a product state (\ref{def:1}), is in an entengled state. 
        \label{def:2}
    \end{definition}
    When a system is in an entengled state it is not possible to characterize the two subsystems
    with the states vector, although the state vector of the composite system is known.

    \subsection{Density operator}
        For a more general treatment, the following representation of states is given:
        \begin{definition}
            The state of quantum system is described by a linear operator, called density
            operator such that:
            \begin{equation*}
                \pmb{\Xi}:\Hilbert\to\Hilbert;\ \pmb{\Xi}^\dagger=\pmb{\Xi};\ tr\{\pmb{\Xi}\}=1.
            \end{equation*}
            \label{def:3}
        \end{definition}
        According to the definition \ref{def:3}, the postulates \ref{post:3}, \ref{post:4},
        \ref{post:5} can be reformulate as following.
        \begin{equation}
            \mathbb{P}(\lambda_i)=tr\{\pmb{\Xi}\mathcal{P}_i\}
            \label{post:3.1}
        \end{equation}
        \begin{equation}
            \pmb{\Xi}'=\frac{\mathcal{P}_i\pmb{\Xi}\mathcal{P}_i^\dagger}
            {tr\{\mathcal{P}_i\pmb{\Xi}\mathcal{P}_i^\dagger\}}
            \label{post:4.1}
        \end{equation}
        \begin{equation}
            \pmb{\Xi}(t)=\mathcal{U}\pmb{\Xi}(t_0)\mathcal{U}^\dagger
            \label{post:5.1}
        \end{equation}