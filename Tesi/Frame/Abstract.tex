% Abstract

\chapter*{Abstract}
    Quantum mechanics is opening doors to new ways of information processing and transmitting.
    The ability to deeply understand the essence of matter will lead, in the foreseeable future, to design
    and implement innovative technologies with a remarkable increase in performance.
    That is a very significant fact if we consider the increasingly computerization and necessity to be
    connected in our society.
    Communication will play a key role in this scenario, so it is essential to study and optimize an
    efficient and reliable communication system.

    The aim of this thesis is analyzing the performance of an optimal binary quantum discriminator, in
    presence of information encoded with photon added quantum states.
    Firstly, the cardinal principles of the quantum mechanics will be introduced (Chapter 2), followed
    by a brief rewiev of some useful tools aimed at characterizing a quantum state-based
    communication system.
    The third chapter will concern a brief presentation of the quantum communication modalities
    (quantum modulation) and some key aspects in the recognition of information (quantum state
    discriminator). In this context, the optimal discriminator for binary systems will be introduced, that
    is the one which minimizes the distribution error probability (DEP) in the recognition of the
    transmitted symbol.
    Finally, the behaviour of the latter will be analyzed in the presence of non-Gaussian states of photon
    added coherent states (PACSs) and photon added squeezed states (PASSs) typologies, with the goal
    of researching the optimal parameters configuration that minimizes the error probabilty.