% Introduction

\chapter{Introduction}
    La meccanica quantistica sta aprendo le porte a nuovi modi di comunicare ed elaborare le 
    informazioni. La capacità di comprendere a fondo l'essenza della materia potrà portare, 
    in un futuro molto vicino, a realizzare e progettare tecnologie di prossima generazione
    con prestazioni notevolmente migliori, in grado di soddisfare le esigenze di un mondo sempre
    più informatizzato e connesso.
    Le comunicazioni in tutto ciò rivestiranno un ruolo chiave; studiare ed ottimizzare un sistema
    di comunicazione efficiente e affidabile risulterà essenziale.

    L'obiettivo di questa tesi è quello di analizzare le prestazioni di un optimal binary quantum discriminator 
    in presenza di informazioni codificate con stati quantistici Photon Added. 
    Verranno presentati innanzitutto i principi cardine della meccanica quantistica (capitolo 2), 
    seguiti da una breve rassegna di alcuni strumenti utili ai fini di 
    caratterizzare un sistema di comunicazione basato sull'utilizzo di stati quantistici.
    Il terzo capitolo riguarderà una rapida presentazione delle modalità comunicazione quantistiche (quantum modulation)
    e degli aspetti chiave nel riconoscimento dell'informazione (Quantum state discrimination). In questo contesto 
    verrà presentato il discriminatore ottimo per binary systems; ovvero quello che minimizza la distribution error
    probability (DEP) nel riconoscimento del simbolo trasmesso. 
    Verrà infine analizzato il comportamento di quest'ultimo in presenza di non-Gaussian states di tipo photon
    added coherent states (PACSs) e photon added squeezed states (PASSs), col fine di ricercare la configurazione
    ottima di parametri che minimizzi la probabilità d'errore.