\section{Postulates}
    Like every phisics theory, quantum mechanics is builded from few 
    essential postulates.
    In this section are briefly introduced the six Dirac-Von Newman 
    postulates of Quantum Mechanics \cite{quantumMec_Dirac,quantumMec_Neumann}.
    
    \subsection{First postulate}
    \begin{postulate}[State Representation]
        The state of an isolated quantum system is represented by a complex unitary 
        vector in an Hilbert space:
        \begin{equation*}
            \ket{\psi} \in \Hilbert
        \end{equation*}
        The space of possible states of the system is called state space and it is a
        separable complex Hilbert space.
        \label{post:1}
    \end{postulate}
    \begin{observation*}
        Differently from the classical physics, in quantum mechanics the concept
        of state of system is introduced. In classical mechanics a system is 
        described by his observables, like position or four-wheeled.
    \end{observation*}
    
    \subsection{Second postulate}
    \begin{postulate}[Observables]
        Every observables of the system is represented by an Hermitian operator
        acting on the state space:
        \begin{equation*}
            \mathcal{M}:\Hilbert\to\Hilbert
        \end{equation*}
        The outcomes of the measurement can only be one of the eigenvalue of the 
        operator $\mathcal{M}$.
        \label{post:2}
    \end{postulate}
    \begin{observation*}
        The possible outcomes of the measurement are real number because $\mathcal{M}$
        is self-andjoint. 
    \end{observation*}

    \subsection{Third postulate}
    \begin{postulate}[Born's Rule]
        The probability to get the measurement $\lambda_i$ from the observable 
        $\mathcal{M}$ in the system in state $\ket{\psi}$ is:
        \begin{equation*}
            %\mathbb{P}(\lambda_i)=\braket{\psi}{\lambda_i}\braket{\lambda_i}{\psi}
            \mathbb{P}(\lambda_i)=\bra{\psi}\mathcal{P}_i\ket{\psi}
        \end{equation*}
        where $\bra{\psi}$ is the correspondent vector of $\ket{\psi}$ in the 
        dual space of $\Hilbert$ and where $\mathcal{P}_i$ is the projection operator
        of $\lambda_i$ in the correspondent space.
        \label{post:3}
    \end{postulate}

    \subsection{Fourth postulate}
    \begin{postulate}[Wavefunction Collapse]
        The state after measurement of $\lambda_i$ is $\mathcal{P}_i\ket{\psi}$ (with the
        necessary normalization):
        \begin{equation*}
            \ket{\psi'}=\frac{\mathcal{P}_i\ket{\psi}}{\bra{\psi}\mathcal{P}_i\ket{\psi}}.
        \end{equation*}
        \label{post:4}
    \end{postulate}

    \subsection{Fifth postulate}
    \begin{postulate}[Time Evolution]
        The time evolution of an isolated quantum system is given by an unitary operator
        $\mathcal{U}$:
        \begin{equation*}
            \ket{\psi(t)}=\mathcal{U}(t_0,t)\ket{\psi(t_0)}.
        \end{equation*}
        \label{post:5}
    \end{postulate}
    \begin{observation*}[Time dependent Shrodinger Equation]
        From postulate \ref{post:5}, it is possible to obtain the time dependent Shrodinger Equation:
        \begin{equation*}
            i\hbar\partialderivative{}{t}\ket{\psi(t)}=H(t)\ket{\psi(t)}
        \end{equation*}
        where $H(t)$ is the Hemiltonian matrix.
    \end{observation*}

    \subsection{Sixth postulate}
    \begin{postulate}[Composite System]
        The state space of a system composed of $\Hilbert_1$ and $\Hilbert_2$ is given by
        \begin{equation*}
            \Hilbert=\Hilbert_1\otimes\Hilbert_2.
        \end{equation*}
        \label{post:6}
    \end{postulate}
