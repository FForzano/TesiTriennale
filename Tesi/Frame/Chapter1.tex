\chapter{Quantum Mechanics Abstract}
    In this chapter, a bief overview of quantum mechanics postulates and
    of the notation used in this thesis is given. The target of that is 
    to explain to the reader the essential concept [...]

    \section{Postulates}
        Like every phisics theory, quantum mechanics is builded from few 
        essential postulates.
        In this section are briefly introduced the six Dirac-Von Newman 
        postulates of Quantum Mechanics \cite{quantumMec_Dirac}\cite{quantumMec_Neumann}.
        
        \subsection{First Postulate}
        \begin{postulate}[State Representation]
            The state of an isolated quantum system is represented by a complex unitary 
            vector in an Hilbert space:
            \begin{equation*}
                \ket{\psi} \in \Hilbert
            \end{equation*}
            The space of possible states of the system is called state space and it is a
            separable complex Hilbert space.
            \label{post:1}
        \end{postulate}
        \begin{observation*}
            Differently from the classical physics, in quantum mechanics the concept
            of state of system is introduced. In classical mechanics a system is 
            described by his observables, like position or four-wheeled.
        \end{observation*}
        
        \subsection{Second Postulate}
        \begin{postulate}[Observables]
            Every observables of the system is represented by an Hermitian operator
            acting on the state space:
            \begin{equation*}
                \mathcal{M}:\Hilbert\to\Hilbert
            \end{equation*}
            The outcomes of the measurement can only be one of the eigenvalue of the 
            operator $\mathcal{M}$.
            \label{post:2}
        \end{postulate}
        \begin{observation*}
            The possible outcomes of the measurement are real number because $\mathcal{M}$
            is self-andjoint. 
        \end{observation*}

        \subsection{Third Postulate}
        \begin{postulate}[Born's Rule]
            The probability to get the measurement $\lambda_i$ from the observable 
            $\mathcal{M}$ in the system in state $\ket{\psi}$ is:
            \begin{equation*}
                %\mathbb{P}(\lambda_i)=\braket{\psi}{\lambda_i}\braket{\lambda_i}{\psi}
                \mathbb{P}(\lambda_i)=\bra{\psi}\mathcal{P}_i\ket{\psi}
            \end{equation*}
            where $\bra{\psi}$ is the correspondent vector of $\ket{\psi}$ in the 
            dual space of $\Hilbert$ and where $\mathcal{P}_i$ is the projection operator
            of $\lambda_i$ in the correspondent space.
            \label{post:3}
        \end{postulate}

        \subsection{Fourth Postulate}
        \begin{postulate}[Wavefunction Collapse]
            The state after measurement of $\lambda_i$ is $\mathcal{P}_i\ket{\psi}$ (with the
            necessary normalization):
            \begin{equation*}
                \ket{\psi'}=\frac{\mathcal{P}_i\ket{\psi}}{\bra{\psi}\mathcal{P}_i\ket{\psi}}.
            \end{equation*}
            \label{post:4}
        \end{postulate}

        \subsection{Fifth Postulate}
        \begin{postulate}[Time Evolution]
            The time evolution of an isolated quantum system is given by an unitary operator
            $\mathcal{U}$:
            \begin{equation*}
                \ket{\psi(t)}=\mathcal{U}(t_0,t)\ket{\psi(t_0)}.
            \end{equation*}
            \label{post:5}
        \end{postulate}
        \begin{observation*}[Time dependent Shrodinger Equation]
            From postulate \ref{post:5}, is possible to found the time dependent Shrodinger Equation:
            \begin{equation*}
                i\hbar\partialderivative{}{t}\ket{\psi(t)}=H(t)\ket{\psi(t)}
            \end{equation*}
            where $H(t)$ is the Hemiltonian matrix.
        \end{observation*}

        \subsection{Sixth Postulate}
        \begin{postulate}[Composite System]
            The state space of a system composite from $\Hilbert_1$ and $\Hilbert_2$ is given by
            \begin{equation*}
                \Hilbert=\Hilbert_1\otimes\Hilbert_2.
            \end{equation*}
            \label{post:6}
        \end{postulate}

    \section{Combining Systems}
    The last postulate \ref{post:6} has very important consequences for composite system. It is possible
    to descrive two tipe of combined systems:

        \begin{definition}[Product States]
            A state $\ket{\psi}\in\Hilbert$ with $\Hilbert=\Hilbert_1\otimes\Hilbert_2$ is a
            pure state if exists $\ket{\psi_1}\in\Hilbert_1$ and $\ket{\psi_2}\in\Hilbert_2$
            such that:
            \begin{equation*}
                \ket{\psi}=\ket{\psi_1}\otimes\ket{\psi_2}.
            \end{equation*}
            \label{def:1}
        \end{definition}
        A product state represent two states which do not interact; an operation on one of 
        them don't perturb the other.

        \begin{definition}[Entengled States]
            A system that is not in a product state (\ref{def:1}), is an entengled state. 
            \label{def:2}
        \end{definition}
        When a system is in an entengled state is not possible to charaterize the two subsystems
        with the states vector, as the state vector of the composite system is known.

        \subsection{Density Operator}
            For a more general treatment, the following representation of states is given:
            \begin{definition}
                The state of quantum system is described by a linear operator, called density
                operator such that:
                \begin{equation*}
                    \Xi:\Hilbert\to\Hilbert;\ \Xi^\dagger=\Xi;\ tr\{\Xi\}=1.
                \end{equation*}
                \label{def:3}
            \end{definition}
            According to the definition \ref{def:3}, the postulates \ref{post:3}, \ref{post:4},
            \ref{post:5} can be reformulate as following.
            \begin{equation}
                \mathbb{P}(\lambda_i)=tr\{\Xi\mathcal{P}_i\}
                \label{post:3.1}
            \end{equation}
            \begin{equation}
                \Xi'=\frac{\mathcal{P}_i\Xi\mathcal{P}_i^\dagger}
                {tr\{\mathcal{P}_i\Xi\mathcal{P}_i^\dagger\}}
                \label{post:4.1}
            \end{equation}
            \begin{equation}
                \Xi(t)=\mathcal{U}\Xi(t_0)\mathcal{U}^\dagger
                \label{post:5.1}
            \end{equation}

        \section{Quantized Electromagnetic Field}
            Electromagnetic field is the main means of communication for contemporary
            application. It is important therefor, to give a quantum representation.
            \par In a volume $\mathcal{V}\in\mathbb{R}^3$ classical magnetic field is 
            determinated from Maxwell's equations as a superposition of the cavity modes
            \begin{equation}
                \mathbf{e}(\mathbf{r},t)=\sum_n p_n (t)\mathbf{u}_n (\mathbf{r})
            \end{equation}
