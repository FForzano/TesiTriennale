\section{Quantum Discriminator}
    The problem of quantum state discrimination (QSD) are one of the most important
    aspect of quantum communication. As in classical communication, the ability to 
    distinguish between two or more states, in presence of noise, can be decisive for improve
    the performance of the communication system. Unlike the classical situation however,
    the discrimination can be done using a custom-designed quantum discriminator, overcoming
    the classical physics limits.

    \subsection{Binary quantum state discrimination}
    \begin{figure}[ht]
        \begin{center}
            \includegraphics[width=0.75\textwidth]{fig2.2.pdf}
            \caption{Binary quantum state discriminator.}
            \label{fig:2.2}
        \end{center}
    \end{figure}
    The problem of discrimination between two quantum states is realized, as every measurement
    process \ref{post:2}, using an operator or with a set of operators.
    If the state of system is unknown, as shown in figure \ref{fig:2.2}, there are two hypotheses
    about the state $\pmb{\Xi}$ (the problem is generalises easly for $M$ different states),
    given by:
    \begin{equation}\begin{split}
        H_0 : \pmb{\Xi}=\pmb{\Xi}_0\\
        H_1 : \pmb{\Xi}=\pmb{\Xi}_1
        \label{eq:binHyp}
    \end{split}\end{equation}
    It is necessary a set of two positive-definites operator (POVM):
    \begin{equation}
        \mathcal{P}=\{\pmb{\Pi}_0,\pmb{\Pi}_1\}
    \end{equation}
    for the discrimination process and the probability that the hypothesis $H_j$ is choosen
    if $H_k$ is the right choose, is given by \cite{tesiGuerrini}:
    \begin{equation}
        \mathbb{P}\{H_j|H_k\}=tr\{\pmb{\Xi}_k\pmb{\Pi}_j\}.
    \end{equation}
    The distribution error probability (DEP) in the discrimination process, if $p_0$ and $p_1$ are 
    respectively the probabilty of symbols $0$ and $1$, is so given by
    \begin{equation}
        P_e=1-\left(p_0 tr\{\pmb{\Xi}_0\pmb{\Pi}_0\}+p_1 tr\{\pmb{\Xi}_0\pmb{\Pi}_0\}\right).
    \end{equation}

    \subsection{Optimal discriminator}
    The issue of finding the optimal POVM that minimize was exhaustively discuss  by Helstrom in 
    \cite{helstrom3,helstrom4}. The minimum distribution error probability (MDEP), for a binary 
    communication system, is given by the well-known Helstrom bound
    \begin{equation}
        \breve{P}_e = \frac{1}{2} \left(1-\norm{p_1 \pmb{\Xi}_1 - p_0 \pmb{\Xi}_0}_1 \right),
        \label{eq:HelstromBound}
    \end{equation}
    where $p_0,\ p_1$ are the probability that the states $\pmb{\Xi}_0,\ \pmb{\Xi}_1$ are trasmitted
    and the operator $\norm{\cdot}_1$ represents the trace norm. 
    The MDEP \ref{eq:HelstromBound} is obtained with the following POVM:
    \begin{equation}
        \breve{\pmb{\Pi}}_0 = \sum_{\substack{i \\ \lambda_i<0}} \ket{\lambda_i}\bra{\lambda_i},
    \end{equation}
    \begin{equation*}
        \breve{\pmb{\Pi}}_1 = 1-\breve{\pmb{\Pi}}_0 = 
        \sum_{\substack{i \\ \lambda_i \geq 0}} \ket{\lambda_i}\bra{\lambda_i};
    \end{equation*}
    where $\ket{\lambda_i}$ is the eigenvector of $p_1 \pmb{\Xi}_1 - p_0 \pmb{\Xi}_0$ associated to 
    the eigenvalue $\lambda_i$.
    For pure states, i.e $\pmb{\Xi}_0 = \ket{\psi_0}\bra{\psi_0}$ and $\pmb{\Xi}_1 = \ket{\psi_1}
    \bra{\psi_1}$, the equation \ref{eq:HelstromBound} begin
    \begin{equation}
        \breve{P}_e = \frac{1}{2} \left(1- \sqrt{1-4 p_0 p_1 \absolutevalue{\braket{\psi_0}{\psi_1}}^2}
        \right).
    \end{equation}
    It is possible to observe that, for pure states, the MDEP is equal to $0$ if $\braket{\psi_0}{\psi_1}$,
    that is $\ket{\psi_0}$ and $\ket{\psi_1}$ are orthogonal states.