\begin{frame}
    \frametitle{Quantum Mechanics Abstract}
    \framesubtitle{Postulates}

    \begin{block}{Postulate 1: State Representation}
        The state of an isolated quantum system is represented by a complex unitary 
        vector in an Hilbert space:
        \begin{equation*}
            \ket{\psi} \in \Hilbert
        \end{equation*}
    \end{block}
    %
    \begin{block}{Postulate 2: Observables}
        Every observables of the system is represented by an Hermitian operator
        acting on the state space:
        \begin{equation*}
            \mathcal{M}:\Hilbert\to\Hilbert
        \end{equation*}
    \end{block}
\end{frame}

\begin{frame}
    \frametitle{Quantum Mechanics Abstract}
    \framesubtitle{Postulates}

    \begin{block}{Postulate 3: Born's Rule}
        The probability to get the measurement $\lambda_i$ from the observable 
        $\mathcal{M}$ in the system in state $\ket{\psi}$ is:
        \begin{equation*}
            %\mathbb{P}(\lambda_i)=\braket{\psi}{\lambda_i}\braket{\lambda_i}{\psi}
            \mathbb{P}(\lambda_i)=\bra{\psi}\mathcal{P}_i\ket{\psi}
        \end{equation*}
    \end{block}
    %
    \begin{block}{Postulate 4: Wavefunction Collapse}
        The state after measurement of $\lambda_i$ is $\mathcal{P}_i\ket{\psi}$ (with the
        necessary normalization):
        \begin{equation*}
            \ket{\psi'}=\frac{\mathcal{P}_i\ket{\psi}}{\bra{\psi}\mathcal{P}_i\ket{\psi}}.
        \end{equation*}
    \end{block}
\end{frame}

\begin{frame}
    \frametitle{Quantum Mechanics Abstract}
    \framesubtitle{Postulates}

    \begin{block}{Postulate 5: Time Evolution}
        The time evolution of an isolated quantum system is given by an unitary operator
        $\mathcal{U}$:
        \begin{equation*}
            \ket{\psi(t)}=\mathcal{U}(t_0,t)\ket{\psi(t_0)}.
        \end{equation*}
    \end{block}
    %
    \begin{block}{Postulate 6: Composite System}
        The state space of a system composed of $\Hilbert_1$ and $\Hilbert_2$ is given by
        \begin{equation*}
            \Hilbert=\Hilbert_1\otimes\Hilbert_2.
        \end{equation*}
    \end{block}

\end{frame}