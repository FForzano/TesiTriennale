\section{Continuos-Variables Quantum Systems}
    A quantum system is called a continuous-variable system
    when it has an infinite-dimensional Hilbert space described
    by observables with continuous eigenspectra \cite{ContinuousVar}.
    Continuous-variables systems play a very important role in communications. this
    section presents the key aspects for the representation of this systems.
            
    \subsection{Hilbert space}
        Let consider a single-mode bosonic continuous-variable system, corresponding to a single
        mode radiation of electromagnetic field, i.e. a single mode quantum harmonic oscillator.
        Its space of states is an infinite dimensional Hilbert space $\Hilbert$, and corresponding
        to a pair of bosonic field operators $\{ \pmb{A},\pmb{A}^{\dagger}\}$ called annihilation
        and creation operators \cite{ContinuousVar}.
        The space of state $\Hilbert$ is infinite dimensional because it is spanned by a countable
        basis $\{\ket{n}\}_{n=0}^\infty$ called Fock basis. This basis is composed by the
        eigenstates of the number operator $\pmb{N}$ defined as
        \begin{equation}
            \pmb{N}=\pmb{A}^\dagger \pmb{A},
        \end{equation}
        i.e. 
        \begin{equation*}
            \pmb{N}\ket{n}=n\ket{n}.
        \end{equation*}

        The action over this states of the bosonic operators is determinated by \cite{ContinuousVar}
        \begin{equation}
            \pmb{A}\ket{0}=0;\ \ \pmb{A}\ket{n}=\sqrt{n}\ket{n-1}\ \ (for\ n \geq 1),
        \end{equation}
        and
        \begin{equation*}
            \pmb{A}^\dagger\ket{n}=\sqrt{n+1}\ket{n+1}\ \ (for\ n \geq 0).
        \end{equation*}
        
        
    \subsection{Phase space}
        As seen before in \ref{TheDensOp}, a quantum system can be completely
        described by a density operator $\pmb{\Xi}$ defined in an infinite-dimensional Hilbert space
        $\Hilbert$, and this operator can be expressed by the Fock representation. %(\ref{}).
        Sometimes, however, it is convenient to give another representation of state $\pmb{\Xi}$ by
        means of a complex function introduced by Wigner \cite{Wigner}: the quasi-probability 
        distribution. In this thesis, this representation will be introduced and it will be used to
        classify the possible states.

        \begin{definition}[Quantum characteristic function]
            The s-order characteristic function $\mathcal{\chi}(\xi,s)$, with $\xi,s\in\mathbb{C}$,
            associated to the quantum state $\pmb{\Xi}$ is defined as:
            \begin{equation}
                \mathcal{\chi}(\xi,s)=\exp{\frac{s}{2}\absolutevalue{\xi}^2}
                tr\{\pmb{\Xi}\pmb{D}_\xi\}
            \end{equation}
            where $\pmb{D}_\xi$ is  the displacement operator of parameter $\xi$, defined as:
            \begin{equation}
                \pmb{D}_\xi=\exp{\xi\pmb{A}^\dagger-\xi^*\pmb{A}}.
            \end{equation}
        \end{definition}
        \begin{definition}[Quasi-probability distribution]
            The s-order quasi-probability distribution $W(\alpha,s)$, with $s\in\mathbb{C}$,
            associated to the quantum state $\mathbf{\Xi}$ is given by:
            \begin{equation*}
                W(\alpha,s)=\frac{1}{\pi^2}\int_{\mathbb{R}^2} 
                \mathcal{\chi}(\xi,s)e^{\alpha\xi^*-\alpha^*\xi}d\xi^2.
            \end{equation*}

        \end{definition}
        The quasi-probability distribution, for $s=0$ ($W(\alpha)=W(\alpha,0)$) is called 
        Wigner W-function.