\chapter{Quantum Mechanics Abstract}
    In this chapter, a bief overview of quantum mechanics postulates and
    of the notation used in this thesis is given. The target of that is 
    to explain to the reader the essential concept [...]

    \section{Postulates}
        Like every phisics theory, quantum mechanics is builded from few 
        essential postulates.
        In this section are briefly introduced the six Dirac-Von Newman 
        postulates of Quantum Mechanics.
        
        \subsection{First Postulate}
        \begin{postulate}[State Representation]
            The state of an isolated quantum system is represented by a complex unitary 
            vector in an Hilbert space:
            \begin{equation*}
                \ket{\psi} \in \Hilbert
            \end{equation*}
            The space of possible states of the system is called state space and is a
            separable complex Hilbert space.
            \label{post:1}
        \end{postulate}
        \begin{observation*}
            Differently to the classical physics, in quantum mechanics the concept
            of state of system is introduced. In classical mechanics a system is 
            described by his observables, like position or four-wheeled.
        \end{observation*}
        
        \subsection{Second Postulate}
        \begin{postulate}[Observables]
            Every observables of the system is represented by an Hermitian operator
            acting on the state space:
            \begin{equation*}
                \mathcal{M}:\Hilbert\to\Hilbert
            \end{equation*}
            The outcomes of the measurement can only be one of the eigenvalue of the 
            operator $\mathcal{M}$.
            \label{post:2}
        \end{postulate}
        \begin{observation*}
            The possible outcomes of the measurement are real number because $\mathcal{M}$
            is self-andjoint. 
        \end{observation*}

        \subsection{Third Postulate}
        \begin{postulate}[Born's Rule]
            The probability to get the measurement $\lambda_i$ from the observable 
            $\mathcal{M}$ in the system in state $\ket{\psi}$ is:
            \begin{equation*}
                \mathbb{P}(\lambda_i)=\braket{\psi}{\lambda_i}\braket{\lambda_i}{\psi}
            \end{equation*}
            where $\bra{\psi}$ is the correspondent vector of $\ket{\psi}$ in the 
            dual space of $\Hilbert$ and $\ket{\lambda_i}$ is the eigenvector correspontent
            to the eigenvalue $\lambda_i$.\\
            Equivalently, it is possible to write:
            \begin{equation*}
                \mathbb{P}(\lambda_i)=\bra{\psi}\mathcal{P_i}\ket{\psi}
            \end{equation*}
            where $\mathcal{P_i}$ is the projection operator corresponding to $\lambda_i$.
            \label{post:3}
        \end{postulate}

        \subsection{Fourth Postulate}
        \begin{postulate}[Wavefunction Collapse]

            \label{post:4}
        \end{postulate}