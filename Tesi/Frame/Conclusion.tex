% Conclusioni

\chapter{Conclusion}
    The aim of this thesis is analyzing the perfomance of a quantum optimal discriminator in the 
    presence of non-gaussian states. This perfomance was evaluated in terms of error probability 
    relating to symbol recognition. 
    In the first place, we have presented the quantum mechanics postulates, formulated by Dirac and 
    Von Neumann and generalized with the use of density operators.
    Then, we have described the quantum modulation and quantum states discrimination concepts. 
    Finally, we have analyzed some systems perfomance in terms of minimum distribution error 
    probability (MDEP).
    All evaluations were made assuming the absence of effects associated to the communication channel, 
    then supposing that the recieved state does not present any difference compared to the emitted one.
    
    This thesis highlights the fact that the use of non-gaussian photon added states instead of gaussian 
    states in OOK systems can ameliorate the QSD.
    In particular, the combination of squeezing and photon addition (PASSs) turns out extremely effective.
    Instead, in BPSK quantum systems, the photon addition effect manifests itself as negative.

    The obtained results can be significantly important in a quantum communication system design, 
    allowing us to make the best of this phisic theory potential. 