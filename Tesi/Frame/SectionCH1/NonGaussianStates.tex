\section{Non-Gaussian States}
    A state that does not fulfill the definition \ref{def:Gaussian} is a non-Gaussian state.
    An important and useful for communications class of non-Gaussian states, are the photon 
    added states, examined in this thesis. Lastly will be mentioned another type of non-Gaussian
    state: the photon subtracted state.
    
    \subsection{Photon added states}
        \label{PAS}
        The photon added state $\pmb{\Xi}^{(1)}$, obtained from the quantum state $\pmb{\Xi}$,
        is given by:
        \begin{equation}
            \pmb{\Xi}^{(1)}=\frac{\pmb{A}^\dagger\pmb{\Xi}\pmb{A}}
            {tr\{\pmb{A}^\dagger\pmb{\Xi}\pmb{A}\}}.
        \end{equation}
        The name \emph{photon addition} could lead to believe that the mean photon number of the 
        photon added state is icreased by one compared to the previous non photon added state.
        However, that is incorrect.
        In general, its mean number of photons could be the same, more or less than the starting state.
        Only if $\pmb{\Xi}=\ket{n}\bra{n}$, i.e $\pmb{\Xi}$ is the density operator corresponding to
        the Fock state $\ket{n}$, the result of the photon addition is a state with one more photon.

        Logically, the photon added state $\pmb{\Xi}^{(k)}$ (with $k$ photon addition) is given by
        \begin{equation}
            \pmb{\Xi}^{(k)}=\frac{(\pmb{A}^\dagger)^k\pmb{\Xi}\pmb{A}^k}
            {tr\{(\pmb{A}^\dagger)^k\pmb{\Xi}\pmb{A}^k\}}.
        \end{equation}
        The Fock representation of a photon added state, can be obteined as:
        \begin{equation}
            \pmb{\Xi}^{(k)}=\frac{\tilde{\pmb{\Xi}}^{(k)}}{tr\{\tilde{\pmb{\Xi}}^{(k)}\}}
        \end{equation}
        and
        \begin{equation*}
            \bra{n}\tilde{\pmb{\Xi}}^{(k)}\ket{m}=
            \begin{cases}
                \sqrt{\frac{n!m!}{(n-k)!(m-k)!}}\bra{n-k}\pmb{\Xi}\ket{m-k}\ if\ n,m \geq k\\
                0\ otherwise.
            \end{cases}
        \end{equation*}
        The Wigner W-function of a photon added state is not Gaussian (\ref{def:Gaussian}).

        \paragraph{Noisy photon added coherent states}\mbox{}\\
        If $\pmb{\Xi}$ is a noisy coherent state of amplitude $\mu$ $(\pmb{\Xi}=\pmb{\Xi}_{th}(\mu))$,
        the photon added state $\pmb{\Xi}_{th}^{(k)}(\mu)$ is called noisy photon added coherent state
        (noisy PACS).
        The Fock representation can be given in closed form by \cite{PACSDisc}
        \begin{equation}
            \bra{n}\pmb{\Xi}_{th}^{(k)}(\mu)\ket{m}=
            \begin{cases}
                c_{n,m}^{(k)}\ \ for\ both\ n,m \geq k\\
                0\ \ otherwise
            \end{cases}
        \end{equation}
        where
        \begin{equation*}
            c_{n,m}^{(k)}=\frac{(1-v)^{k+1}e^{-(1-v)\absolutevalue{\mu}^2}}{v^k}
            \sqrt{n!}{m!}\binom{m}{k} v^n \left[\left(1-v\right)\mu^*\right]^{m-n}
            \frac{L^{m-n}_{n-k} \left( \frac{-(1-v)\absolutevalue{\mu}^2}{v} \right)}
            {L_k \left( -\absolutevalue{\mu}^2 (1-v) \right)}.
        \end{equation*}
        The Wigner W-function is given by:
        \begin{equation}
            W(\alpha)=\frac{(-1)^k}{(2\bar{n}+1)^k}\frac{L_k \left( 
                \frac{\absolutevalue{2\alpha(\bar{n}+1)-\mu}^2}{(2\bar{n}+1)(\bar{n}+1)} \right)}
                {L_k} \left(-\frac{\absolutevalue{\mu}^2}{\bar{n}+1} \right) W_{th}(\alpha)
        \end{equation}
        where $W_{th}(\alpha)$ is the Wigner W-function of a noisy coherent state \ref{WignerNCS}.
        %
        \begin{figure}[tbp]
            \begin{center}
            \begin{subfigure}{0.5\textwidth}
                
                \caption{Noisy coherent state}
            \end{subfigure}
            \begin{subfigure}{0.5\textwidth}
                
                \caption{noisy PACS}
            \end{subfigure}
            \caption{Comparison between Noisy coherent state and noisy PACS Wigner W-function\\
            $\bar{n}=10^{-2}$ and $k=2$.}
            \label{fig:WignerPACS}
            \end{center}
        \end{figure}
        In figure \ref{fig:WignerPACS} are plotted the Wigner W-function of a noisy coherent
        state and a noisy PACS, for $\bar{n}=10^{-2}$ and $k=2$. It is evident that the Wigner 
        W-function of the photon added state is not Gaussian.
        
        \paragraph{Noisy photon added squeezed states}\mbox{}\\
        if $\pmb{\Xi}$ is a noisy squeezed state with amplitude $\mu$ and squeezing factor 
        $\zeta$ $(\pmb{\Xi}=\pmb{\Xi}_{th}(\mu,\zeta))$, the photon added state $\pmb{\Xi}_{th}^{(k)}
        (\mu,\zeta)$ is called noisy photon added squeezed state (PASS).

    \subsection{Photon subtracted states}