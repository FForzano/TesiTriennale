\section{Preliminaries on quantum mechanics}
    For understand the important results about the communication with continuos
    states, it is essential to give a brief introduction about the main aspects 
    of quantum mechanics theory.
    This theory is based on a solid mathematic framework presented in this section.
    It is impossible to discuss about quantum mechanics without its mathematical 
    formalism.
    
    \subsection{Postulates}
    Like every phisics theory, quantum mechanics is builded from few 
    essential postulates.
    In this section are briefly introduced the six Dirac-Von Newman 
    postulates of Quantum Mechanics \cite{quantumMec_Dirac,quantumMec_Neumann}.

    \begin{postulate}[State Representation]
        The state of an isolated quantum system is represented by a complex unitary 
        vector $\ket{\psi}$ in an Hilbert space $\Hilbert$:
        \begin{equation*}
            \ket{\psi} \in \Hilbert
        \end{equation*}
        The space of possible states of the system is called state space and it is a
        separable complex Hilbert space.
        \label{post:1}
    \end{postulate}
    \begin{observation*}
        Differently from the classical physics, in quantum mechanics the concept
        of state of system is introduced. In classical mechanics a system is 
        described by his observables, like position or four-wheeled.
    \end{observation*}
    
    \begin{postulate}[Observables]
        Every observables of the system is represented by an Hermitian operator $\mathcal{M}$
        acting on the state space:
        \begin{equation*}
            \mathcal{M}:\Hilbert\to\Hilbert
        \end{equation*}
        The outcomes of the measurement can only be one of the eigenvalue of the 
        operator $\mathcal{M}$.
        \label{post:2}
    \end{postulate}
    \begin{observation*}
        The possible outcomes of the measurement are real number because $\mathcal{M}$
        is self-andjoint. 
    \end{observation*}

    \begin{postulate}[Born's Rule]
        The probability to get the measurement $\lambda_i$ from the observable 
        $\mathcal{M}$ in the system in state $\ket{\psi}$ is:
        \begin{equation*}
            %\mathbb{P}(\lambda_i)=\braket{\psi}{\lambda_i}\braket{\lambda_i}{\psi}
            \mathbb{P}(\lambda_i)=\bra{\psi}\mathcal{P}_i\ket{\psi}
        \end{equation*}
        where $\bra{\psi}$ is the correspondent vector of $\ket{\psi}$ in the 
        dual space of $\Hilbert$ and where $\mathcal{P}_i$ is the projection operator
        of $\lambda_i$ in the correspondent space.
        \label{post:3}
    \end{postulate}

    \begin{postulate}[Wavefunction Collapse]
        The state $\ket{\psi'}$ after measurement of $\lambda_i$ is $\mathcal{P}_i\ket{\psi}$ (with the
        necessary normalization):
        \begin{equation*}
            \ket{\psi'}=\frac{\mathcal{P}_i\ket{\psi}}{\bra{\psi}\mathcal{P}_i\ket{\psi}}.
        \end{equation*}
        \label{post:4}
    \end{postulate}

    \begin{postulate}[Time Evolution]
        The time evolution $\ket{\psi(t)}$ of an isolated quantum system is given by an unitary operator
        $\mathcal{U}$:
        \begin{equation*}
            \ket{\psi(t)}=\mathcal{U}(t_0,t)\ket{\psi(t_0)}.
        \end{equation*}
        \label{post:5}
    \end{postulate}
    \begin{observation*}[Time dependent Shrodinger Equation]
        From postulate \ref{post:5}, it is possible to obtain the time dependent Shrodinger Equation:
        \begin{equation*}
            i\hbar\partialderivative{}{t}\ket{\psi(t)}=H(t)\ket{\psi(t)}
        \end{equation*}
        where $H(t)$ is the Hemiltonian matrix, $\hbar$ is the reduxed Planck's constant and $i=\sqrt{-1}$
        is the immaginary unit.
    \end{observation*}

    \begin{postulate}[Composite System]
        The state space $\Hilbert$ of a system composed of $\Hilbert_1$ and $\Hilbert_2$ is given by
        \begin{equation*}
            \Hilbert=\Hilbert_1\otimes\Hilbert_2.
        \end{equation*}
        \label{post:6}
    \end{postulate}

    \subsection{The density operator}
    \label{TheDensOp}
    The last postulate \ref{post:6} has very important consequences for composite system. It is possible
    to describe two tipes of combined systems:

    \begin{definition}[Product states]
        A state $\ket{\psi}\in\Hilbert$ with $\Hilbert=\Hilbert_1\otimes\Hilbert_2$ is a
        pure state if exists $\ket{\psi_1}\in\Hilbert_1$ and $\ket{\psi_2}\in\Hilbert_2$
        such that:
        \begin{equation*}
            \ket{\psi}=\ket{\psi_1}\otimes\ket{\psi_2}.
        \end{equation*}
        \label{def:1}
    \end{definition}
    A product state represents two systems which do not interact; an operation on one of 
    them does not perturb the other.

    \begin{definition}[Entengled states]
        A system that is not in a product state (\ref{def:1}), is in an entengled state. 
        \label{def:2}
    \end{definition}
    When a system is in an entengled state it is not possible to characterize the two subsystems
    with the states vector, although the state vector of the composite system is known.

    \paragraph{Density operator}\mbox{}\\
        For a more general treatment, the following representation of states is given:
        \begin{definition}
            The state of quantum system is described by a linear operator $\pmb{\Xi}$, called density
            operator such that:
            \begin{equation*}
                \pmb{\Xi}:\Hilbert\to\Hilbert;\ \pmb{\Xi}^\dagger=\pmb{\Xi};\ tr\{\pmb{\Xi}\}=1.
            \end{equation*}
            \label{def:3}
        \end{definition}
        According to the definition \ref{def:3}, the postulates \ref{post:3}, \ref{post:4},
        \ref{post:5} can be reformulate as following.
        \begin{equation}
            \mathbb{P}(\lambda_i)=tr\{\pmb{\Xi}\mathcal{P}_i\}
            \label{post:3.1}
        \end{equation}
        \begin{equation}
            \pmb{\Xi}'=\frac{\mathcal{P}_i\pmb{\Xi}\mathcal{P}_i^\dagger}
            {tr\{\mathcal{P}_i\pmb{\Xi}\mathcal{P}_i^\dagger\}}
            \label{post:4.1}
        \end{equation}
        \begin{equation}
            \pmb{\Xi}(t)=\mathcal{U}\pmb{\Xi}(t_0)\mathcal{U}^\dagger
            \label{post:5.1}
        \end{equation}