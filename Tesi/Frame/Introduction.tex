\chapter{Introduction}
    The information theory, born from the publications of Claude Shannon in 
    the last fifty years of the XX century, has allowed us to study the communication
    world. The technological evolution of the last years has as its basis the concept of
    communication.
    Quantum mechanics can be the key enabler for the next generation communication systems.
    The possibility to engineer the properties of quantum systems is essential for the design
    of an optimized quantum communication systems.

    Quantum technologies can be classified into two categories (\cite{tesiGuerrini}): discrete 
    variables systems (DVs) and continuous variables systems (CVs). DV technologies are based
    on discrete quantum states, such as qubits, and are the quantum equivalents of digital
    signals. CV technologies are based on continuous values quantum states, such as 
    coherent states, and are the equivalents of analog signals.
    The use of CVs offers the possibility to use the existing classical network
    infrastructure by just adapting the apparatuses in the network nodes; and, on the other
    side, CVs states are more easy to generate and manage \cite{tesiGuerrini}.
    For this reasons, in this thesis we will describe and analyze the second category of quantum
    communication system.

    \section{Quantum Communication}
        With quantum communication we mean the task of transferring classical or quantum information
        (\cite{GueChiWinCon:C20,GueChiWinCon:C19,GueChiCon:C18,ChiConWin:J20}) from one place to another
        one, using quantum technologies.
        The use of quantum states for the representation of informations allows us to overcome the 
        limits of classical communication systems.
        In particular, the use of non-Gaussian photon-added states, as it will be shown in this thesis,
        can improves significantly the performance of communication systems. 
        The use of photon-added states like photon-added coherent states (PACSs) and photon-added 
        squeezed states (PASSs) is importat too for the possibility to easily generate these states
        by using linear and non-linear optical devices.
        The main objective of this thesis is therfore to analyze the performance of quantum 
        communication systems using PACSs 
        \cite{PAPACSDisc,GueChiWinCon:C20,GueChiWinCon:C19,GueChiCon:C18} and PASSs.

    \section{Objectives}
        The goal of this thesis, as mentioned before, is to analyze the performance of quantum 
        communication systems using PACSs and PASS. The main content of this thesis are:
        \begin{description}
            \item[Chapter 2:] A brief introduction of the quantum theory for CVs with particular emphasis on
                    the characterization of non-Gaussian photon-added states;
            \item[Chapter 3:] The description of a quantum communication system;
            \item[Chapter 4:] The performance analysis of quantum systems using PACSs and PASSs.
        \end{description}