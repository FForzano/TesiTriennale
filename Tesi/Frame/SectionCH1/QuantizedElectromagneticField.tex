\section{Quantized Electromagnetic Field}
    Electromagnetic field is the main means of communication for contemporary
    application. It is important therefor, to give a quantum representation.
    In this section the representation of quantized electromagnetic field is 
    initially given, so the Fock representation of a state is introduced.
            
    \subsection{Classical electromagnetic field}
        In a volume $\mathcal{V}\in\mathbb{R}^3$ classical electromagnetic field is 
        determinated from Maxwell's equations as a superposition of the cavity modes
        (\cite{tesiGuerrini} quoting \cite{quantumRad_Louissel,quantumOptic_Mandel}).
        Electric field is given by the well-known expression:
        \begin{equation}
            \mathbf{e}(\mathbf{r},t)=\sum_n p_n (t)\mathbf{u}_n (\mathbf{r})
            \label{eq:CEF.1}
        \end{equation}
        where
        \begin{equation*}
            \mathbf{u}_n (\mathbf{r})=\mathbf{u}_{n0}\ e^{i\mathbf{k}_n \cdot \mathbf{r}}
        \end{equation*}
        and $\mathbf{u}_{n0}$ is determinated by the initial condition.
        The corresponding magnetic field is determinated by:
        \begin{equation}
            \mathbf{h}(\mathbf{r},t)=\sum_n q_n (t)\nabla\times\mathbf{u}_n (\mathbf{r})
            \label{eq:CEF.2}
        \end{equation}
        and
        \begin{equation}
            p_n (t)=\derivative{q_n (t)}{t} .
            \label{eq:CEF.3}
        \end{equation}
        The Hemiltonian associated to the n-th mode is given by
        \begin{equation}
            H_n=\frac{1}{2}[p_n^2(t)+\omega_n^2q_n^2(t)].
            \label{eq:CEF.4}
        \end{equation}
        Equivalently, it is possible to define the complex variable $a_n(t)$ as
        \begin{equation}
            a_n(t)=\frac{\omega_nq_n(t)+ip_n(t)}{\sqrt{2\hbar\omega_n}}
            \label{eq:CEF.5}
        \end{equation}
        and, using \ref{eq:CEF.5} in \ref{eq:CEF.4}, it is possible to obtain the following
        expression of the Hemiltonian:
        \begin{equation}
            H_n=\hbar\omega_n\absolutevalue{a_n(t)}^2.
            \label{eq_CEF.6}
        \end{equation}

    \subsection{Quantized electromagnetic field}
        The quantization of electromagnetic field is obtained replacing the two quantities 
        $p_n(t)$ and $q_n(t)$ with the Hermitian operators 
        $\mathbf{P}_n(t),\ \mathbf{Q}_n(t):\Hilbert_n\to\Hilbert_n$ and by imposing the following
        commutation conditions (\cite{tesiGuerrini} quoting \cite{quantumRad_Louissel,quantumOptic_Mandel}):
        \begin{equation}
            \commutator{\mathbf{Q}_n}{\mathbf{P}_m}=i\hbar\delta_{n,m}\mathbf{I}
        \end{equation}
        \begin{equation}
            \commutator{\mathbf{Q}_n}{\mathbf{Q}_m}=0
        \end{equation}
        \begin{equation}
            \commutator{\mathbf{P}_n}{\mathbf{P}_m}=0.
        \end{equation}
        Defining the annihilation operator $\mathbf{A}_n$ as
        \begin{equation}
            \mathbf{A}_n(t)=\frac{\omega_n\mathbf{Q}_n(t)+i\mathbf{P}_n(t)}{\sqrt{2\hbar\omega_n}}
            \label{eq:QEF.1}
        \end{equation} 
        and the adjoint of $\mathbf{A}_n$, the creation operator $\mathbf{A}_n^\dagger$ as
        \begin{equation}
            \mathbf{A}_n(t)=\frac{\omega_n\mathbf{Q}_n(t)-i\mathbf{P}_n(t)}{\sqrt{2\hbar\omega_n}}
            \label{eq:QEF.2}
        \end{equation}
        it is possible to describe the Hemiltonian of the system as
        \begin{equation}
            H_n=\hbar\omega_n\mathbf{A}_n^\dagger\mathbf{A}_n.
            \label{eq:QEF.3}
        \end{equation}

    \subsection{Fock states}
        In a single mode cavity, it is possible to define the number operator $\mathbf{N}$ as
        \begin{equation}
            \mathbf{N}=\mathbf{A}^\dagger \mathbf{A}.
        \end{equation}
        Single mode Fock states are the eigenvector of $N$, i.e the solution of equation:
        \begin{equation}
            \mathbf{N}\ket{n}=n\ket{n}.
        \end{equation}
        The Fock state $\ket{n}$ represent the quantum state with exactly n photons.
        It is important to evidence that the set of all Fock states forms an orthonormal basis
        of the Hilbert space $\Hilbert$, so every state $\Xi$ can be expressed as
        \begin{equation}
            \mathbf{\Xi} = \sum_{n,m} c_{n,m}\ket{n}\bra{m}
            \label{eq:QEF.4}
        \end{equation}
        with
        \begin{equation*}
            c_{n,m}=\bra{n}\mathbf{\Xi}\ket{m}.
        \end{equation*}

        Using the representation in Fock basis, it is possible to charaterize some type of quantum 
        states of the quantum electromagnetic field. In the following section the states studied 
        are briefly described.