\section{Gaussian States}
    \label{def:Gaussian}
    Gaussian quantum states are an important class of quantum states of continuous-variables systems.
    They are defined as (\cite{tesiGuerrini} quoting \cite{Gaussian1,Gaussian2,Gaussian3,Gaussian4,Gaussian5}):
    \begin{definition}[Gaussian state]
        A quantum state $\Operator{\varXi_G}$ is a Gaussian state if its Wigner W-function $W_G(\alpha)$
        is Gaussian, i.e
        \begin{equation}
            W_G(\alpha)=\frac{1}{\pi\sqrt{\det{\check{\Operator{C}_0}}}}\exp{
                -\frac{1}{2}(\check{\Vector{\alpha}}-\check{\Vector{\mu}})^H
                \check{\Operator{C}_0}^{-1}(\check{\Vector{\alpha}}-\check{\Vector{\mu}})}.
        \end{equation}
        where $\check{\Vector{\mu}} $ is the augmented displacement vector, and $\check{\Operator{C}}_0$ 
        is the augmented covariance matrix. 
    \end{definition}
    
    We remark that if $\Vector{\mu} \in \mathbb{R}^2$ is the 
    displacement vector and $\Operator{C}_0$ is the covariance matrix, the augmented displacement vector
    and covariance matrix are given by the following transformation:
    \begin{equation}
        \begin{split}
            \check{\Vector{\mu}} &= \frac{1}{\sqrt{2}} \Operator{J} \Vector{\mu}\\
            \check{\Operator{C}}_0 &= \frac{1}{2} \Operator{J} \Operator{C}_0 \Operator{J}^H
        \end{split}
    \end{equation}
    where
    \begin{equation*}
        \Operator{J} = 
        \begin{bmatrix}
            1 & i\\
            1 & -i
        \end{bmatrix}.
    \end{equation*}
    %
    Two important types of Gaussian states will be analyzed now: the coherent state and the
    squeezed state. For each one of these states is presented the noisy version too.

    \subsection{Coherent state}
        A coherent state is the state of a quantum armonic oscillator of amplitude $\mu$.
        It is defined (\cite{tesiGuerrini} seen \cite{CohSt_Glauber,CohSt_Glauber2}) as the eigenvector $\ket{\mu}$ of $\pmb{A}$ 
        associated to the eigenvalue $\mu$; i.e
        \begin{equation}
            \Operator{A}\ket{\mu}=\mu\ket{\mu}.
        \end{equation}
        It is possible to obtain a coherent state of parameter $\mu$, appliying the displacement
        operator to the ground state:
        \begin{equation}
            \ket{\mu}=\Operator{D}_\mu\ket{0}.
        \end{equation}
        As mentioned before, it is possible to characterize a state with the Fock representation
        and, equivalently, with the Wigner W-function. The last one is given, for a coherent state,
        by \cite{QuantumNoise}:
        \begin{equation}
            W(\alpha)=\frac{2}{\pi}\exp{-2\absolutevalue{\alpha-\mu}^2}.
            \label{eq:WignerCoh}
        \end{equation}
        It is easy to proof that $W(\alpha)$ is gaussian, with $\check{\Vector{\mu}}=[\mu\ \mu^*]^T$ and
        \begin{equation*}
            \check{\Operator{C}}_0=\frac{1}{2}\Operator{I}.
        \end{equation*} 
        The Fock representation is given by \cite{Dowling}:
        \begin{equation}
            \ket{\mu}=e^{-\frac{\absolutevalue{\mu}}{2}^2}\sum_{n=0}^{\infty}
            \frac{\mu^n}{\sqrt{n}}\ket{n}.
        \end{equation}

        \paragraph{Noisy coherent states}\mbox{} \\
        \label{par:NoisycohState}
        It is possible to characterize the state of a noisy armonic oscillator introducing
        the thermal state, i.e the state of an electromagnetic cavity in thermal equilibrium.
        The Fock representation of the thermal state $\pmb{\Xi}_{th}$ is given by \cite{tesiGuerrini}
        \begin{equation}
            \Operator{\varXi}_{\mathrm{th}}=(1-v)\sum_{n=0}^{\infty}v^n\ket{n}\bra{n}
        \end{equation}
        where
        \begin{equation*}
            v=\frac{\bar{n}}{\bar{n}+1}
        \end{equation*}
        and $\bar{n}$ is the well-known Plank distribution
        \begin{equation*}
            \bar{n}=\left(\exp{\frac{\hbar\omega}{k_B T}}-1\right)^{-1}.
            \label{eq:nbar}
        \end{equation*}

        A noisy coherent states $\Operator{\varXi}_{\mathrm{th}}(\mu)$ of parameter $\mu$ can be obtained by 
        appling the displacement operator $\Operator{D}_\mu$ to the thermal state $\Operator{\varXi}_{\mathrm{th}}$,
        as follow:
        \begin{equation}
            \Operator{\varXi}_{\mathrm{th}}(\mu)=\Operator{D}_\mu^\dagger \Operator{\varXi}_{\mathrm{th}} \Operator{D}_\mu.
        \end{equation}
        The Wigner W-function is given by \cite{QuantumNoise}
        \begin{equation}
            W_{th}(\alpha)=\frac{1}{\pi(\bar{n}+\frac{1}{2})}\exp{-\frac{\absolutevalue{\alpha-\mu}^2}
            {\bar{n}+\frac{1}{2}}}
            \label{WignerNCS}
        \end{equation}
        and it can be proved that it is a Gaussian function with $\check{\Vector{\mu}}=[\mu\ \mu^*]^T$
        and
        \begin{equation*}
            \check{\Operator{C}}_0=\left(\bar{n}+\frac{1}{2}\right)\Operator{I}.
        \end{equation*}
        The Fock representation is given by
        \begin{equation}
            \bra{n}\Operator{\Xi}_{\mathrm{th}}(\mu)\ket{m}=(1-v)e^{-(1-v)\absolutevalue{\mu}^2}\sqrt{\frac{n!}{m!}}
            v^n[(1-v)\mu^*]^{m-n}L_n^{m-n}\left(\frac{-(1-v)^2\absolutevalue{\mu}^2}{v}\right)
            \label{eq:FRCS}
        \end{equation}

    \subsection{Squeezed state}
        \label{squeezedStates}
        A squeezed state with amplitude $\mu$ and squeezing parameter $\zeta$, is a defined as 
        \cite{tesiGuerrini,YuenRadField,QMnoiseInterf}
        \begin{equation}
            \ket*{\mu,\zeta}=\Operator{D}_\mu\Operator{S}_\zeta\ket{0}
        \end{equation}
        where $\Operator{S}_\zeta$ is the squeezing operator, defined as
        \begin{equation}
            \Operator{S}_\zeta=\exp{\frac{1}{2}\left(\zeta\left(\Operator{A}^\dagger\right)^2+
            \zeta^*\Operator{A}^2\right)}.
        \end{equation}
        It can be proven that a squeezed state is a Gaussian state with $\check{\Vector{\mu}}=[\mu\ \mu^*]^T$
        and
        \begin{equation*}
            \check{\Operator{C}}_0=\frac{1}{2}
            \begin{bmatrix}
                \cosh(2r) && \sinh(2r)e^{-i\phi}\\
                \sinh(2r)e^{-i\phi} && \cosh(2r)
            \end{bmatrix}
        \end{equation*}
        with $\zeta=re^{i\phi}$.
        The Wigner W-function of a squeezed state, differently from the one of a coherent state, has not a 
        circular symmetry.

        \paragraph{Noisy squeezed states}\mbox{} \\
        The representation of a noisy squeezed state $\Operator{\varXi}_{\mathrm{th}}(\mu,\zeta)$ is obtained,
        similarly to a noisy coherent state, as:
        \begin{equation}
            \Operator{\varXi}_{\mathrm{th}}(\mu,\zeta)=\Operator{D}_\mu\Operator{S}_\zeta\Operator{\varXi}_{\mathrm{th}}
            \Operator{S}_\zeta^\dagger\Operator{D}_\mu^\dagger.
        \end{equation}
        The Gaussian Wigner W-function is obtained with $\check{\Vector{\mu}}=[\mu\ \mu^*]^T$ and
        \begin{equation}
            \check{\pmb{C}}_0=\left(\bar{n}+\frac{1}{2}\right)
            \begin{bmatrix}
                \cosh(2r) && \sinh(2r)e^{-i\phi}\\
                \sinh(2r)e^{-i\phi} && \cosh(2r)
            \end{bmatrix}.
            \label{eq:WignerSS}
        \end{equation}
        The Fock representation is given by \cite{MarMar_1993}
        \begin{equation}\begin{split}
            \bra{n}\Operator{\varXi}_{\mathrm{th}}(\mu,\zeta)\ket{m}=\frac{\pi Q(0)}{(n!m!)^{1/2}}
            \sum_{k=0}^{min(n,m)} k! \binom{n}{k}\binom{m}{k} \tilde{A}^k\left(\frac{1}{2}
            \tilde{B}\right)^{(n-k)/2}\\ \left(\frac{1}{2}\tilde{B}^*\right)^{(m-k)/2}
            H_{n-k}((2\tilde{B})^{-1/2}\tilde{C}) H_{m-k}((2\tilde{B}^*)^{-1/2}\tilde{C}^*) 
            \label{FR_NSS}
        \end{split}\end{equation}
        where $H_n(x)$ is the Hermite polynomial with parameter $n$,
        \begin{equation}
            \begin{split}
            Q(0) &= \frac{1}{\pi}[(1+A)^2-\absolutevalue{B}^2]^{-1/2}\exp{-\frac{(1+A)\absolutevalue{C}^2
            +\frac{1}{2}[B(C^*)^2+B^*C^2]}{(1+A)^2-\absolutevalue{B}^2}}\\
            \tilde{A} &= \frac{A(1+A)-\absolutevalue{B}^2}{(1+A)^2-\absolutevalue{B}^2}\\
            \tilde{B} &= \frac{B}{(1+A)^2-\absolutevalue{B}^2}\\
            \tilde{C} &= \frac{(1+A)C+BC^*}{(1+A)^2-\absolutevalue{B}^2}.
            \end{split}
        \end{equation}
        The parameter $A,B$ and $C$ are defined as:
        \begin{equation}
            \begin{split}
            A&=\bar{n}+(2\bar{n}+1)(\sinh(r))^2\\
            B&=-(2\bar{n}+1)e^{i\phi}\sinh(r)\cosh(r)\\
            C&=\mu.
            \end{split}
        \end{equation}